\documentclass[11pt]{beamer}
\usepackage{listings} % Include the listings-package
\usepackage[T1]{fontenc}
\usepackage[utf8]{inputenc}
\usepackage[english]{babel}
\usepackage{amsmath}
\usepackage{amssymb, amsfonts, latexsym, cancel}
\usepackage{float}
\usepackage{graphicx}
\usepackage{epstopdf}
\usepackage{subfigure}
\usepackage{hyperref}
%\usepackage{authblk}
\usepackage{blindtext}
\usepackage{booktabs} % Allows the use of \toprule, 
\usepackage{filecontents}
\usepackage{courier} %% Sets font for listing as Courier.
\usepackage{listings}
%\usepackage{listings, xcolor}
\lstset{
tabsize = 2, %% set tab space width
showstringspaces = false, %% prevent space marking in strings, string is defined as the text that is generally printed directly to the console
numbers = left, %% display line numbers on the left
commentstyle = \color{green}, %% set comment color
keywordstyle = \color{blue}, %% set keyword color
stringstyle = \color{red}, %% set string color
rulecolor = \color{black}, %% set frame color to avoid being affected by text color
basicstyle = \small \ttfamily , %% set listing font and size
breaklines = true, %% enable line breaking
numberstyle = \tiny,
}
\usepackage{caption}
\DeclareCaptionFont{white}{\color{white}}
\DeclareCaptionFormat{listing}{\colorbox{gray}{\parbox{\textwidth}{#1#2#3}}}
\captionsetup[lstlisting]{format=listing,labelfont=white,textfont=white}
\definecolor{urlColor}{rgb}{0.06, 0.3, 0.57}
\definecolor{linkColor}{rgb}{0.57, 0.0, 0.04}
\definecolor{fileColor}{rgb}{0.0, 0.26, 0.26}
\hypersetup{
    colorlinks=true,
    linkcolor=linkColor,
    filecolor=fileColor,      
    urlcolor=urlColor,
}
\urlstyle{same}
\setbeamercovered{transparent}
%\usetheme{Boadilla}
\usetheme{CambridgeUS}
%\usetheme{Berkeley}
%\usetheme{Warsaw}
%\usetheme{Madrid}

\title[Presentación]{\bf\Huge Diseño de Software Interactivo Educativo Inclusivo}
\subtitle{Iteración Humano Computador 2020}

\author[IHC2020]
{
	Marco Antonio Mendoza Rodríguez \\
    Luis Fernando Quispe Sanomamani\\
    Christopher Brad Del Castillo Montoya\\
    Fatima Gigi Rojas Carhuas\\
}
\institute[UNSA]
{
\inst{1}% 
Universidad Nacional de San Agustin\\
Escuela Profesional de Ingenieria de Sistemas
}

\date[2020-10-05]{\scriptsize{2020-10-05}}
%\logo{\includegraphics[width=3.0cm]{img/logo_unsa.jpg}}
\titlegraphic{\includegraphics[width=1.0cm]{img/unsa.jpg}}

\begin{document}

\begin{frame}
\titlepage
\end{frame}

\begin{frame}
\frametitle{Content}
\tableofcontents
\end{frame}

\section{Educación Inclusiva}
\begin{frame}
\frametitle{Educación Inclusiva}
\begin{itemize}
\item Es aquella que se organiza con la intención de responder a todas las necesidades e intereses, se adapta a los ritmos de aprendizaje y tiene en cuenta las potencialidades del alumnado

\end{itemize}
\end{frame}


\section{Discapacidad}
\begin{frame}
\frametitle{Discapacidad}
De acuerdo con la clasificación Internacional del funcionamiento, de la discapacidad y de la salud, presenta en 2001, las personas con discapacidad “Son aquellas que tienen una o más deficiencias físicas, mentales, intelectuales o sensoriales y que la interactuar con distintos ambientes del entorno social pueden impedir su participación plena y efectiva en igualdad de condiciones a las demás”.
\end{frame}


\section{Tipos de Problemas de Aprendizaje}
\begin{frame}
\frametitle{Tipos de Problemas de Aprendizaje}
\begin{itemize}
\item Dislexia
\item Disgrafia
\item Discalculía
\item Discapacidad de Memoria
\item TDHA
\item Trastorno del espectro autista
\item Discapacidad Intelectual
\end{itemize}
\end{frame}


\section{USAER}
\begin{frame}
\frametitle{USAER (Unidad de Servicio de Apoyo en la Educación Regular)}
Programa educativo del Ministerio de Educación de México que a través de un equipo multidisciplinario (analista, psicólogos, maestros, tecnólogos) que ayuda al maestro titular que sirve para poder identificar necesidades especiales de aprendizaje en los niños
\end{frame}

\section{¿Cómo ayudar a los maestros?}
\begin{frame}
\frametitle{¿Cómo ayudar a los maestros?}
Acelerando la capacitación de los mismos ya que en estos tiempos modernos los niños ya están muy familiarizados con la tecnología actual como son: los dispositivos modernos, el manejo de tecnologías de información
\end{frame}

\section{Adecuación de los Recursos Educativos}
\begin{frame}
\frametitle{Adecuación de los recursos educativos}
En una escuela inclusiva, deben diseñarse recursos que estén adaptados de manera que puedan ser utilizados y captados por todo el alumnado.
Deben contener estímulos multisensoriales y permitir la accesibilidad a todo el alumnado
Las aplicaciones interactivas pueden mejorar la competencia básica en los niños con dificultades de aprendizaje.

\end{frame}

\section{Analisis}
\begin{frame}
\frametitle{Análisis}
\begin{itemize}
\item Con una estadistica descriptiva en un analisis de 200 estudiantes y lo que sobresale son problemas de lenguaje, transtornos de atencion.
\item Los maestros cuentan con una gran cantidad de aplicaciones educativas en linea y se pueden acomodar a base de los intereses de los maestros sobre repositorios como hablidades matematicas, escritura, etc. 
Además de ello se puede acomodar a sus necesidades ya que se puede conectar desde cualquier dispositivo.
\end{itemize}
\end{frame}
\begin{frame}
\frametitle{Ecosistema de apoyo}

{\includegraphics[width=12cm]{img/eco.png}}


\end{frame}




\begin{frame}
\frametitle{Patrones educacionales}
\begin{itemize}
\item Un Patron es la solucion a un problema recurente
\item  Un problema recurrente que presentan los alumnos es en la identificacion de silabas y una forma que el maestro noe ste encima de el todo el tiempo es por medio de la aplicacion en que el niño pueda usarla cuando el quiera y de manera ludica.
\item Otro es hacer simples frases que trate el mal entendido de frases.

\end{itemize}
\end{frame}
\begin{frame}
\frametitle{Modelos de procesos para desarrolar aplicaciones educativas}
\begin{itemize}
\item Alisis de requerimientos
\item Diseño
\item Desarrollo
\item Lanzamiento


\end{itemize}
\end{frame}

\begin{frame}
\section{Fases}
\frametitle{Fases}
{\includegraphics[width=5.0cm]{img/d.png}}
{\includegraphics[width=5.0cm]{img/n.png}}
{\includegraphics[width=5.0cm]{img/u.png}}
{\includegraphics[width=5.0cm]{img/p.png}}
\end{frame}

\begin{frame}
{\includegraphics[width=5.0cm]{img/g.png}}
{\includegraphics[width=5.0cm]{img/a.png}}
\end{frame}

\begin{frame}
\section{Capacitación docente}
\frametitle{Capacitación docente}
\begin{itemize}
\item Los docentes recibieron charlas educativas e instructivas sobre el manejo de diversas tecnologías como las redes sociales las cuales son  empleadas para un óptimo desarrollo del aprendizaje en el estudiante .
\end{itemize}
\end{frame}

\begin{frame}
\section{Resultados de las fases y delas pruebas de usabilidad}
\frametitle{Resultados de las fases y delas pruebas de usabilidad}
\begin{itemize}
\item Se aplicaron a 8 niños divididos en grupo A con quienes se trabajo un año y grupo B donde reforzo la atención teniendo como resultado una buena aceptación y mejorando su aprendizaje.
{\includegraphics[width=5.0cm]{img/f.png}}

\end{itemize}
\end{frame}

\begin{frame}
\section{Mas aplicaciones}
\frametitle{Mas aplicaciones}
\begin{itemize}
\item Kinect para personas con discapacidad auditiva
{\includegraphics[width=5.0cm]{img/l.png}}
\item Audiolibros para personas con debilidad visual
{\includegraphics[width=5.0cm]{img/v.png}}
\end{itemize}
\end{frame}

\begin{frame}

\frametitle{}
\textbf{¿Cual es la estrategia que se utiliza para el desarrollo de aplicaciones para niños con habilidades especiales?}
\begin{itemize}
\item Acompañar el desarrollo de la aplicación  con el progreso y maduración del estudiante
\item Se requiere un equipo multidisciplinario, no solo tecnologos, sino también pedagogos , psicologos, entre otros
\end{itemize}


\end{frame}

\begin{frame}
\frametitle{}

\textbf{¿En que conviene utilizarse las pruebas de usabilidad?}
\begin{itemize}
\item Acompañar el desarrollo de la aplicación  con el progreso y maduración del estudiante
\item Se requiere un equipo multidisciplinario, no solo tecnologos, sino también pedagogos , psicologos, entre otros
\end{itemize}


\end{frame}


\begin{frame}
    

\frametitle{Conclusiones}
\begin{itemize}
\item Aportar una metodología de atención a niños con problemas de aprendizaje utilizando aplicaciones educativas
\item Registrar lo requerimientos en base a la atención de estudiantes que enfrentan problemas con el aprendizaje
\item Capacitar a los maestros en la utilización de estos recursos
\end{itemize}

\end{frame}

\section{Referencias}
% Marco de referencias
\begin{frame}
\frametitle{Referencias}
\begin{itemize}
\item  \href{https://recursos.portaleducoas.org/sites/default/files/VE14.061.pdf}{Diseño de recurso Didactico}
\item  \href{http://sedici.unlp.edu.ar/bitstream/handle/10915/31833/Documento_completo.pdf?sequence=1&isAllowed=yy}{TCIs para educacion inclusiva}
\item  \href{https://www.redalyc.org/pdf/688/68830444003.pdf}{El uso de un software educativo para promover el aprecio por la diversidad en alumnos de primaria}

\end{itemize}
\end{frame}
\end{document}
